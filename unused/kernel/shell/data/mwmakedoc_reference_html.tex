%
% LaTeX List Header (with html package) for the mwmakedoc command
%
%----------------------------------------------------------------------------
% This file is part of the MegaWave2 system macros. 
% MegaWave2 is a "soft-publication" for the scientific community. It has
% been developed for research purposes and it comes without any warranty.
% The last version is available at http://www.cmla.ens-cachan.fr/Cmla/Megawave
% CMLA, Ecole Normale Superieure de Cachan, 61 av. du President Wilson,
%       94235 Cachan cedex, France. Email: megawave@cmla.ens-cachan.fr 
%-----------------------------------------------------------------------------

\markboth{Reference \hfill MegaWave2 User's Modules Library \hfill Reference \hspace{1cm}}{Reference \hfill MegaWave2 User's Modules Library \hfill Reference \hspace{1cm}}

\part{Reference}
\label{part_reference}

This part gives descriptions of all user's modules in alphabetical order
of groups, and then in alphabetical order of function names (as in the
\hyperref{list given here}{list given by Part~}{}{part_list}). 

For quick reference of a module for which you already know its name or for which 
you may describe its category, refer to the \hyperref[pageref]{index}{index page~}{}{print_index}. 

You may also find the module your are looking for by considering the 
\hyperref[pageref]{bibliography}{bibliography page~}{}{bibliography} :
for each reference, the list of modules making citation to this reference is given.
This bibliography is therefore the main entry for researchers 
who want to reproduce experiments reported in the cited articles, if they have been
performed using MegaWave2 (in such a case names of module's authors generally match
- at least partially - names of article's authors), following the so-called {\em reproducible
research paradigm}.

\newpage

