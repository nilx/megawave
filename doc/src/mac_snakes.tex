This module implements the Kimmel and Bruckstein snake model
\cite{KB2} for contour detection 
(or optimization). Its goal is to maximize with respect with 
$\gamma$ (an arclength parameterized curve) the average contrast
$$E(\gamma) =\frac 1 {L(\gamma)}\int_{0}^{L(\gamma)} g\left(\frac{}{}\!\!
Du(\gamma(s))\cdot n(s)\right)ds,$$
where $g$ is an increasing scalar function, $L(\gamma)$ the length of
the curve $\gamma$ and $n(s)$ its normal vector (that is, orthogonal
to $\gamma'(s)$). Such a model has several advantages compared to
the classical Kass-Witjin-Terzopoulos model \cite{KWT} and
the more elaborate model of geodesic active contours from 
Caselles, Kimmel and Sapiro \cite{CKS}. 
The use of a normal contrast (instead
of a gradient norm) is an improvement to the model of Fua
and Leclerc \cite{FL}.

This module only implements the case
$g(s)=|s|^{power}$, but it could be easily adapted to more general 
functions $g$. The influence of the
choice of $g$ is discussed in \cite{DMM}.
The default is $g(s)=|s|$, but powers smaller than 1 could be preferred
in order to reduce even more the sensitivity to the image contrast.

The numerical scheme is based on a gradient descent (with step {\it step})
associated to a discrete version of $E(\gamma)$ (see the description 
below, and \cite{DMM} for a complete discussion and full details).
An input Fimage ($u$) and an initial set {\it in}
of curves (Dlists) should be provided
to the module, which outputs a new (final) set of curves {\it out}
after {\it niter} iterations (\verb+-n+ option). Intermediate
states can be monitored by printing the energy evolution (with
\verb+-v+ option) or visually with the \verb+-V+ option followed by
a zoom factor (try 2 for example). Note that the number of points of
each curve is kept constant during the evolution, 
so that an appropriate (fine enough) sampling 
of each initial curve is needed: rough polygons will not do in general!

\vskip 8mm

\centerline {\bf \large Practical Use}

\vskip 6mm

Due to their sensitivity to initial conditions, snakes models should
be used for interactive contour optimization rather than for contour
detection. To obtain good results, one should

\begin{itemize}
\item choose an initial contour close enough to the optimal solution seeked;
\item used a smoothed initial image (obtained for example with 
\verb+fsepconvol+ or \verb+amss+) in order to avoid local maxima
of the snakes model;
\item use a correctly sampled initial contour (if used, \verb+readpoly+ 
should be followed by \verb+gass+);
\item select an appropriate value for the gradent step (\verb+-s+ option),
which can easily go from 0.01 to more than 1000.
\end{itemize}

An example is given by the following sequence of commands:

\begin{verbatim}
circle -r 20 -n 100 c
fkzrt c c 1 0 115 170
fkview -s -b cimage c &
fsepconvol -g 1 cimage u
mac_snakes -V 2 -s 1 -n 3000 u c out
\end{verbatim}

For an interactive choice of the initial
contour, the first two lines may be replaced by 
\begin{verbatim}
readpoly cimage c
gass -e 2 c c
\end{verbatim}
(type 'q' after drawing manually the initial contour).

%\vskip 8mm
\newpage

\centerline {\bf \large Numerical Scheme}

\vskip 6mm

For a non-Euclidean
parameterization $\gamma(p):[a,b]\to I\!\!R^2$, the energy we want to
maximize writes
\begin{equation}
F(\gamma) = \frac{\int_a^b g\left(Du.\frac{\gamma'(p)^\perp}{|\gamma'(p)|}\right) \,|\gamma'(p)|\, dp}
{\int_a^b |\gamma'(p)|\, dp}.
\label{energy_cont.equ}
\end{equation}
Rather than writing the Euler equation for (\ref{energy_cont.equ})
and {\it then} discretizing it, we discretize the energy and
compute its exact derivative with respect to the discrete curve.
Let us suppose that the snake is represented by a polygonal curve
$M_1..M_n$ (either closed or with fixed endpoints). For the curve
length, we can take
$$L = \sum_i \|\Delta_i\|\qquad\mathrm{with}\qquad \Delta_i=M_{i+1}-M_i.$$
Now the discrete energy can be written $F = \frac EL$,
with
$$E=\sum_i g(t_i)\|\Delta_i\|,$$
$$t_i = w_i \cdot \frac{\Delta_i}{\|\Delta_i\|},\quad
w_i = {D u}^{\bot}(\Omega_i),\quad \mathrm{and}\quad \Omega_i =
\frac{M_i+M_{i+1}}2.$$ Differentiating $E$ with respect to $M_k$,
we obtain
$$\nabla_{M_k} F = \frac 1L \left( \frac{}{}\!\!
\nabla_{M_k} E- F\nabla_{M_k} L\right)$$
with
$$\nabla_{M_k} L= \frac{\Delta_{k-1}}{\|\Delta_{k-1}\|}
- \frac{\Delta_k}{\|\Delta_k\|},$$
$$\nabla_{M_k} E= v_k+v_{k-1}+z_{k-1}-z_k,$$
$$v_i = \frac 12
g'(t_i)D(({D u}^{\bot})^T)(\Omega_i) \Delta_i,$$
$$z_i = g'(t_i) w_i +h(t_i)\frac{\Delta_i}{\|\Delta_i\|}$$
and $h(t)=g(t)-tg'(t)$. Note that
$$D(({D u}^{\bot})^T)
= D ( -u_y\;\; u_x) = \left(\begin{array}{cc}
-u_{xy} & u_{xx} \\
-u_{yy} & u_{xy}
\end{array}\right).$$
Numerically, we compute $Du$ at integer points
with a $3\times 3$ finite differences scheme,
and $D^2 u$ with the same scheme applied to the computed components of
$Du$. This introduces a slight smoothing of the derivatives, which
counterbalances a little the strong locality of the snake model.
These estimations at integer points are then extended to the whole
plane using a bilinear interpolation.

\medskip

To compute the evolution of the snake, we use a two-steps iterative
scheme.
\begin{enumerate}
\item The first step consists in a reparameterization of the snake
according to arc length. It can be justified in several ways~:
aside from bringing stability to the scheme, it guarantees a {\it
geometric} evolution of the curve, it ensures an homogeneous
estimate of the energy, and it prevents singularities to appear
too easily. However, we do not prevent self-intersections of the
curve.
\item The second step is simply a gradient evolution with a fixed step.
If $(M_i^n)_i$ represents the (polygonal) snake at iteration $n$ and
$(\tilde M_i^n)_i$ its renormalized version after step 1, then we set
$$M_i^{n+1} = \tilde M_i^n + step \cdot  \nabla_{\tilde M_i^n} F.$$
\end{enumerate}

\begin{thebibliography}{1}

\bibitem{CKS}
V. Caselles, R. Kimmel and G. Sapiro,
``Geodesic active contours'',
{\it International Journal of Computer Vision} 1:22, pp. 61-79, 1997.

\bibitem{DMM}
A. Desolneux, L. Moisan, J.-M. Morel,
``Variational Snakes Theory'',
in {\it Geometric Level Set Methods in Imaging, Vision and Graphics}, 
S. Osher and N. Paragios edts, Springer Verlag, march 2003.

\bibitem{FL}
P. Fua, Y.G. Leclerc,
``Model driven edge detection'',
{\it Machine Vision and Applications} 3, pp. 45-56, 1990.

\bibitem{KWT}
M. Kass, A. Witkin, D. Terzopoulos,
``Snakes: active contour models'',
{\it IEEE Int. Comp. Vis. Conf.}, vol. 777, 1987.

\bibitem{KB2}
R. Kimmel, A. Bruckstein,
``On edge detection, edge integration and geometric active contours'',
{\it Proceedings of Int. Symposium on Mathematical Morphology} (ISMM 2002), 
Sydney, april 2002.

\end{thebibliography}


