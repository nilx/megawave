This module performs arithmetic encoding of symbols contained in the Input 
fimage according to the algorithm described in \cite{kn:wnc}. 

The -s option enables to specify that only the first Size symbols should 
be encoded. 

If the -n option is selected, then symbols in input are supposed to be 
integer in floating point representation ranging from 0 to NSymbol - 1. 
If it is not activated, then a first pass is made on Input to search 
for the minimal alphabet. 

The -c option enables to specify the capacity of the empirical source 
distribution histogram used for arithmetic encoding (see \cite{kn:wnc}). 
The default value is 100 times the size of alphabet. This only makes sense 
when adaptive encoding is performed (i.e. the -h option is not selected). 

The -p option specifies that predictive encoding should be performed. 

If the -h option is activated, then the histogram contained in the fsignal 
Histo is used as the fixed source distribution histogram.  
If this option is not activated, then adaptive encoding is performed, 
meaning that the source distribution histogram is flat at the beginning, 
and is updated after the encoding of each symbol so that it is 
adapted to the sequence of symbols encoded in the past. 

If the -H option is selected, then a header is inserted at the beginning 
of input with information on size of Input alphabet and predictive encoding. 

Notice that the 2D structure of the input fimage is not used in this module. 
The symbols are read in the natural order in the ``gray'' field of the fimage. 


\begin{thebibliography}{999999}

\bibitem[WNC]{kn:wnc} I.H. Witten, R.M. Neal, J.G. Cleary, {\em ``Arithmetic 
coding for data compression'', } Comm. of the ACM, Vol.~30, no~6, pp.520--540, 
(1987). 

\end{thebibliography}
