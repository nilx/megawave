This macro compresses a RGB color image using a vector quantization 
algorithm applied to the orthogonal/biorthogonal wavelet coefficients. 
It is based on the {\em fwvq} module (which is applied separately 
to each channel), and is simpler to use. 

It works exactly in the same way that the {\em Fwvq} macro for graylevel 
images, except that there is a possibility to use different codebooks 
for the red, green and blue channels. If the optional argument 
{\em codebook\_green} and {\em codebook\_blue} are activated, 
then the codebook sets whose prefix is {\em codebook\_red} are used 
to compress the red channel, the codebook sets whose prefix is 
{\em codebook\_green} are used to compress the green channel and 
the codebook sets whose prefix is {\em codebook\_blue} are used 
to compress the blue channel. Otherwise the same codebook sets whose prefix 
is {\em codebook\_red} are used to compress all three red, green and 
blue channels. 

The target rate (if -R option is selected) is the same for red, green and 
blue channels. Notice that {\em Rate} is a target bit rate per point {\bf and} 
per channel. Three separate compressed files are generated 
(*\_{\em Rate}r.comp, *\_{\em Rate}g.comp and *\_{\em Rate}b.comp 
if the input image file is *.rim or *.img). 
The quantized image is put in *\_{\em Rate}q.rim. 

Like in {\em Fwvq} the dimensions of the image should have a minimum 
of factor 2 in their decomposition in prime numbers. If not then a part of the 
image as large as possible is extracted and having the required 
number of factor 2. 

Notice that {\em codebook\_red}, {\em codebook\_green} 
and {\em codebook\_blue} are prefixes for one channel codebook sets 
which may be simply generated with {\em Fwlbg\_adap} macro. 
One must however first extract the channel images of the training color 
images using the {\em cfcolor2channels} module.
