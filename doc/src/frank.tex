The repartition function of a Fimage $u$ 
defined on the domain $\Omega=\{0..M-1\}\times\{0..N-1\}$ is
the nondecreasing function 
$$H_u(t) = \frac{|\{(x,y) \in \Omega;u(x,y)\leq t\}|}{|\Omega|}.$$
As well, its lower repartition may be defined by
$$H^-_u(t) = \frac{|\{(x,y) \in \Omega;u(x,y)< t\}|}{|\Omega|}.$$
This module computes an average repartition function
$$g(t) = w\cdot H_u(t)+ (1-w)H^-_u(t),$$
usually with $w=0.5$, but this weight may be changed with the \verb+-w+
option. If the \verb+-c+ option us selected, then $g$ is multiplied
by $256$ (so that $\lim_\infty g=256$ instead of $\lim_\infty g=1$). 
This function $g$ may serve as a contrast change to produce
the rank image $rank=g(u)$, performing the so-called histogram equalization
of $u$, which is optimal in a general framework
(see \cite{moisan} for more details). 

If requested by the \verb+-g+ option, the function $g$ is returned
as the 2-Flist $(t_i,g(t_i))$, where $(t_i)$ is the (increasing) sorted
sequence of all values taken by $u$.

\begin{thebibliography}{1}

\bibitem{moisan}
L. Moisan, {\it Modelling in image processing},
course notes, DEA Math\'ematiques, Vision, Apprentissage (MVA),
Ecole Normale Sup\'erieure de Cachan, 2002.

\end{thebibliography}
