\Name{mw\_change\_morpho\_set}{Define a morpho set, if not already defined}
\Summary{
Morpho\_set mw\_change\_morpho\_set(ms)

Morpho\_set ms;
}
\Description
This function returns a \mset\ structure if the input \verb+ms = NULL+.
It is provided despite the \\
\verb+mw_new_morpho_set()+ function for
global coherence with other memory types.

The function \verb+mw_change_morpho_set+ returns \Null\ if not enough memory is available to allocate the structure. 
Your code should check this return value to send an error message in the \Null\ case, and do appropriate statement.

Since the MegaWave2 compiler allocates structures for input and output 
objects (see \volI), this function is normally used only for internal objects.
Do not forget to deallocate the internal structures before the end
of the module, except if they are part of an input or output chain.

\Next
\Example
\begin{verbatim}

Morpho_set ms=NULL; /* Internal use: no Input neither Output of module */
Hsegment seg=NULL;

/* Define a morpho set containing one segment only */

if (!(seg=mw_change_hsegment(seg)) ||
    !(ms=mw_change_morpho_set(ms))) mwerror(FATAL,1,"Not enough memory.\n");
seg->xstart=0; 
seg->xend=200;
seg->y=10;

ms->first_segment=seg;
ms->minvalue=0.0;
ms->maxvalue = 1.0;
ms->area=201;

\end{verbatim}

\newpage %......................................


\Name{mw\_copy\_morpho\_set}{Copy a morpho set into another one}
\Summary{
Morpho\_set mw\_copy\_morpho\_set(in,out)

Morpho\_set in, out;
}
\Description
This function copies the \mset\ \verb+in+ into \verb+out+.
The chain of segments are also duplicated.
The result is put in \verb+out+, which may not be a predefined structure : in case 
of \verb+out=NULL+, the \verb+out+ structure is allocated.

The function \verb+mw_copy_morpho_set+ returns \Null\ if not enough memory is available to perform
the copy, or \verb+out+ elsewhere.
Your code should check this return value to send an error message in the \Null\ case, and do appropriate statement.

\Next
\Example
\begin{verbatim}

Morpho_set in; /* Predefined morpho_set */
Morpho_set out=NULL; 

out=mw_copy_morpho_set(in,out);
if (!out) mwerror(FATAL,1,"Not enough memory.\n");
\end{verbatim}

\newpage %......................................


\Name{mw\_delete\_morpho\_set}{Deallocate a morpho set}
\Summary{
void mw\_delete\_morpho\_set(ms)

Morpho\_set ms;
}
\Description
This function deallocates the \mset\ \verb+ms+, including the
chain of horizontal segments.
You should set \verb+ms = NULL+ after this call since the address pointed
by \verb+ms+ is no longer valid.

\Next
\Example
\begin{verbatim}

Morpho_set ms; /* Internal use: no Input neither Output of module */
Hsegment seg;

/* Define a morpho set containing one segment only */

if (!(seg=mw_new_hsegment()) ||
    !(ms=mw_new_morpho_set())) mwerror(FATAL,1,"Not enough memory.\n");
seg->xstart=0; 
seg->xend=200;
seg->y=10;

ms->first_segment=seg;
ms->minvalue=0.0;
ms->maxvalue = 1.0;
ms->area=201;

/* .
   .
   (statement)
   .
   .
*/

/* Deallocate the morpho_set */
mw_delete_morpho_set(ms);

\end{verbatim}

\newpage %......................................


\Name{mw\_length\_morpho\_set}{Return the number of segments a morpho set contains}
\Summary{
unsigned int mw\_length\_morpho\_set(ms)

Morpho\_set ms;
}
\Description
This function returns the number of segments contained in the input
\verb+ms+.
It returns $0$ if the structure is empty or undefined.

\Next
\Example
\begin{verbatim}

Morpho_set ms=NULL; /* Internal use: no Input neither Output of module */
Hsegment seg=NULL;

/* Define a morpho set containing one segment only */

if (!(seg=mw_change_hsegment(seg)) ||
    !(ms=mw_change_morpho_set(ms))) mwerror(FATAL,1,"Not enough memory.\n");
seg->xstart=0; 
seg->xend=200;
seg->y=10;

ms->first_segment=seg;
ms->minvalue=0.0;
ms->maxvalue = 1.0;
ms->area=201;

/* This will print 1 */
printf("%d",mw_length_morpho_set(ms));
\end{verbatim}

\newpage %......................................


\Name{mw\_new\_morpho\_set}{Create a new morpho set}
\Summary{
Morpho\_set mw\_new\_morpho\_set()

}
\Description
This function returns a new \mset\ structure, or \Null\ if not enough 
memory is available to allocate the structure. 
Your code should check this return value to send an error message in the 
\Null\ case, and do appropriate statement.

The new structure is created with fields set to $0$ or \Null.

\Next
\Example
\begin{verbatim}

Morpho_set ms; /* Internal use: no Input neither Output of module */
Hsegment seg;

/* Define a morpho set containing one segment only */

if (!(seg=mw_new_hsegment()) ||
    !(ms=mw_new_morpho_set())) mwerror(FATAL,1,"Not enough memory.\n");
seg->xstart=0; 
seg->xend=200;
seg->y=10;

ms->first_segment=seg;
ms->minvalue=0.0;
ms->maxvalue = 1.0;
ms->area=201;
\end{verbatim}

\newpage %......................................


