\Name{mw\_change\_hsegment}{Define the hsegment structure, if not defined}
\Summary{
Hsegment mw\_change\_hsegment(seg)

Hsegment seg;
}
\Description
This function returns a \hsegment\ structure if the input \verb+seg = NULL+.
It is provided despite the \\
\verb+mw_new_hsegment()+ function for
global coherence with other memory types.

The function \verb+mw_change_hsegment+ returns \Null\ if not enough memory is available to allocate the structure. 
Your code should check this return value to send an error message in the \Null\ case, and do appropriate statement.

Since the MegaWave2 compiler allocates structures for input and output 
objects (see \volI), this function is normally used only for internal objects.
Do not forget to deallocate the internal structures before the end
of the module, except if they are part of an input or output chain.

\Next
\Example
\begin{verbatim}

Hsegment seg=NULL; /* Internal use: no Input neither Output of module */

/* Define the horizontal segment (0,10)-(200,10) */

seg = mw_change_hsegment(seg);
if (seg == NULL) mwerror(FATAL,1,"Not enough memory.\n");
seg->xstart=0; 
seg->xend=200;
seg->y=10;
\end{verbatim}

\newpage %......................................


\Name{mw\_delete\_hsegment}{Deallocate a chain of horizontal segments}
\Summary{
void mw\_delete\_hsegment(seg)

Hsegment seg;
}
\Description
This function deallocates the chain of horizontal segments starting
from \verb+seg+. Previous segments are not deallocated.
You should set \verb+seg = NULL+ after this call since the address pointed
by \verb+seg+ is no longer valid.

\Next
\Example
\begin{verbatim}
Hsegment seg0,newseg,oldseg; 
int i;

/* Create a chain of 10 horizontal segments, starting from seg0 */

if (!(seg0=mw_new_hsegment())) mwerror(FATAL,1,"Not enough memory.\n");
seg0->xstart=0; seg0->xend=200; seg0->y=1;
oldseg=seg0;
for (i=2; i<=10; i++)
{
  if (!(newseg=mw_new_hsegment())) mwerror(FATAL,1,"Not enough memory.\n");
  newseg->xstart=0; newseg->xend=200; newseg->y=i;
  newseg->previous=oldseg;
  oldseg->next=newseg;
  oldseg=newseg;
}

/* .
   .
   (statement)
   .
   .
*/

/* Deallocate the chain of segments */
mw_delete_hsegment(seg0);
\end{verbatim}

\newpage %......................................


\Name{mw\_new\_hsegment}{Create a new hsegment structure}
\Summary{
Hsegment mw\_new\_hsegment()

}
\Description
This function returns a new \hsegment\ structure, or \Null\ if not enough 
memory is available to allocate the structure. 
Your code should check this return value to send an error message in the 
\Null\ case, and do appropriate statement.

The new structure is created with fields set to $0$ or \Null.

\Next
\Example
\begin{verbatim}

Hsegment seg; /* Internal use: no Input neither Output of module */

/* Define the horizontal segment (0,10)-(200,10) */

if (!(seg=mw_new_hsegment())) mwerror(FATAL,1,"Not enough memory.\n");
seg->xstart=0; 
seg->xend=200;
seg->y=10;
\end{verbatim}

\newpage %......................................


