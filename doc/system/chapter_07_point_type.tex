\Name{mw\_change\_point\_type}{Define the point\_type structure, if not defined}
\Summary{
Point\_type mw\_change\_point\_type(pt)

Point\_type pt;
}
\Description
This function returns a \pt\ structure if the input \verb+pt = NULL+.
It is provided despite the \\
\verb+mw_new_point_type()+ function for
global coherence with other memory types.

The function \verb+mw_change_point_type+ returns \Null\ if not enough memory is available to allocate the structure. 
Your code should check this return value to send an error message in the \Null\ case, and do appropriate statement.

Since the MegaWave2 compiler allocates structures for input and output 
objects (see \volI), this function is normally used only for internal objects.
Do not forget to deallocate the internal structures before the end
of the module, except if they are part of an input or output chain.

\Next
\Example
\begin{verbatim}

Point_type pt=NULL; /* Internal use: no Input neither Output of module */

/* Define a point type as image border */

pt = mw_change_point_type(pt);
if (pt == NULL) mwerror(FATAL,1,"Not enough memory.\n");
pt->type = 1; /* image border */
\end{verbatim}

\newpage %......................................


\Name{mw\_copy\_point\_type}{Copy all point types starting from the given one}
\Summary{
Point\_type mw\_copy\_point\_type(in,out)

Point\_type in, out;
}
\Description
This function copies the current point type and the next point types contained in the chain defined at the starting point type \verb+in+. 
The result is put in \verb+out+, which
may not be a predefined structure : in case of \verb+out=NULL+, the \verb+out+
structure is allocated.

The function \verb+mw_copy_point_type+ returns \Null\ if not enough memory is available to perform
the copy, or \verb+out+ elsewhere.
Your code should check this return value to send an error message in the \Null\ case, and do appropriate statement.

\Next
\Example
\begin{verbatim}

Point_type in; /* Predefined point */
Point_type out=NULL; 

out=mw_copy_point_type(in,out);
if (!out) mwerror(FATAL,1,"Not enough memory.\n");
\end{verbatim}

\newpage %......................................


\Name{mw\_delete\_point\_type}{Deallocate the point\_type structure}
\Summary{
void mw\_delete\_point\_type(pt)

Point\_type pt;
}
\Description
This function deallocates the \pt\ structures starting from
the given \verb+pt+, including this point itself.
You should set \verb+pt = NULL+ after this call since the address pointed
by \verb+pt+ is no longer valid.
To deallocate a point only and not all the next points of the
chain, just use \verb+free(pt)+.

\Next
\Example
\begin{verbatim}
/* Remove the first point_type of an existing morpho_line */

Morpho_line ll; /* Existing morpho_line (e.g. Input of module) */
Point_type pt;  /* Internal use */

pt = ll->first_type;
ll->first_type=pt->next;
pt->next->previous = NULL;
free(pt);
pt = NULL;

/* Remove all point_type of an existing morpho_line */

mw_delete_point_type(ll->first_type);
\end{verbatim}

\newpage %......................................


\Name{mw\_new\_point\_type}{Create a new point\_type structure}
\Summary{
Point\_type mw\_new\_point\_type();
}
\Description
This function creates a new \pt\ structure.
It returns \Null\ if not enough memory is available to create the structure.
Your code should check this value to send an
error message in the \Null\ case, and do appropriate statement.

Since the MegaWave2 compiler allocates structures for input and output 
objects (see \volI), this function is normally used only for internal objects.
Do not forget to deallocate the internal point structures before the end
of the module, except if they are part of an input or output curve.

\Next
\Example
\begin{verbatim}
/* Insert the point (0,0) with type 1 at the end of an existing morpho_line */

Morpho_line ll;  /* Existing morpho_line (e.g. Input of module) */
Point_curve point,p;  /* Internal use: no Input neither Output of module */
Point_type pt,t;

/* Define the point (0,0) with type 1 */
point = mw_new_point_curve();
if (point == NULL) mwerror(FATAL,1,"Not enough memory.\n");
pt = mw_new_point_type();
if (pt == NULL) mwerror(FATAL,1,"Not enough memory.\n");
point->x = point->y = 0;
pt->type=1;

/* Find the last point of the morpho_line */
p = ll->first_point; t = ll->first_type; 
while (p->next) {p=p->next; t=t->next;}

/* Insert the point */
p->next = point;
t->next = pt;
point->previous = p;
pt->previous = t;

/* Do not deallocate point_curve and point_type or morpho_line will become inconsistent */ 
\end{verbatim}
\newpage %......................................

