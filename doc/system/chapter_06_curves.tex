\Name{mw\_change\_curves}{Define the curves structure, if not defined}
\Summary{
Curves mw\_change\_curves(curves)

Curves curves;
}
\Description
This function returns a Curves structure if the input \verb+curves = NULL+.
It is provided despite the \verb+mw_new_curves+ function for
global coherence with other memory types.

The function \verb+mw_change_curves+ returns \Null\ if not enough memory is available to allocate the structure. 
Your code should check this return value to send an error message in the \Null\ case, and do appropriate statement.

Since the MegaWave2 compiler allocates structures for input and output 
objects (see \volI), this function is normally used only for internal objects.
Do not forget to deallocate the internal curves structures before the end
of the module.

\Next
\Example
\begin{verbatim}
/* Define a curves set to be two pre-defined curves  */

Curves curves=NULL;   /* Internal use: no Input neither Output of module */
Curve curve1,curve2;  /* Pre-defined curves (e.g. inputs of module) */

curves = mw_change_curves(curves);
if (curves == NULL) mwerror(FATAL,1,"Not enough memory.\n");
...
\end{verbatim}
(End of this example as for the \verb+mw_new_curves+ function).


\newpage %......................................


\Name{mw\_delete\_curves}{Deallocate a curves set}
\Summary{
void mw\_delete\_curves(curves)

Curves curves;
}
\Description
This function deallocates all the memory allocated by the curves variable
that is, all the points belonging to all curves into this set, all \curve structures and the \curves structure itself.
You should set \verb+curves = NULL+ after this call since the address pointed
by \verb+curves+ is no longer valid.

\Next
\Example
\begin{verbatim}
Curves curves=NULL;   /* Internal use: no Input neither Output of module */

curves = mw_new_curves();
if (curves == NULL) mwerror(FATAL,1,"Not enough memory.\n");
.
.
.
mw_delete_curves(curves);
\end{verbatim}

\newpage %......................................


\Name{mw\_length\_curves}{Return the number of curves into a curves structure}
\Summary{
unsigned int mw\_length\_curves(cvs);

Curves cvs;
}
\Description
This function returns the number of curves contained in the given
\verb+cvs+. It returns $0$ if the structure is empty.

\Next
\Example
\begin{verbatim}
/* Define a curves set to be two pre-defined curves  */

Curves curves=NULL;   /* Internal use: no Input neither Output of module */
Curve curve1,curve2;  /* Pre-defined curves (e.g. inputs of module) */

curves = mw_new_curves();
if (curves == NULL) mwerror(FATAL,1,"Not enough memory.\n");

curves->first=curve1;
curve1->previous = curve2->next = NULL;
curve1->next = curve2;
curve2->previous = curve1;

/* The length would be 2 */
printf("Length=%d\n",mw_length_curves(curves));

\end{verbatim}
\newpage %......................................

\Name{mw\_new\_curves}{Create a new curves}
\Summary{
Curves mw\_new\_curves();
}
\Description
This function creates a new \curves structure.
It returns \Null\ if not enough memory is available to create the structure.
Your code should check this value to send an
error message in the \Null\ case, and do appropriate statement.

Since the MegaWave2 compiler allocates structures for input and output 
objects (see \volI), this function is normally used only for internal objects.
Do not forget to deallocate the internal structures before the end
of the module.

\Next
\Example
\begin{verbatim}
/* Define a curves set to be two pre-defined curves  */

Curves curves=NULL;   /* Internal use: no Input neither Output of module */
Curve curve1,curve2;  /* Pre-defined curves (e.g. inputs of module) */

curves = mw_new_curves();
if (curves == NULL) mwerror(FATAL,1,"Not enough memory.\n");

curves->first=curve1;
curve1->previous = curve2->next = NULL;
curve1->next = curve2;
curve2->previous = curve1;
\end{verbatim}
\newpage %......................................

\Name{mw\_npoints\_curves}{Return the total number of points a curves structure contains}
\Summary{
unsigned int mw\_npoints\_curves(cvs);

Curves cvs;
}
\Description
This function returns the total number of points contained in the given
\verb+cvs+, that is the sum of \verb+mw_length_curve(cv)+ for all
curves \verb+cv+ contained in \verb+cvs+.
The function returns $0$ if the structure is empty.

\newpage %......................................

