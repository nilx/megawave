\Name{mw\_change\_morpho\_line}{Define a morpho line, if not already defined}
\Summary{
Morpho\_line mw\_change\_morpho\_line(ml)

Morpho\_line ml;
}
\Description
This function returns a \mline\ structure if the input \verb+ml = NULL+.
It is provided despite the \\
\verb+mw_new_morpho_line()+ function for
global coherence with other memory types.

The function \verb+mw_change_morpho_line+ returns \Null\ if not enough memory is available to allocate the structure. 
Your code should check this return value to send an error message in the \Null\ case, and do appropriate statement.

Since the MegaWave2 compiler allocates structures for input and output 
objects (see \volI), this function is normally used only for internal objects.
Do not forget to deallocate the internal structures before the end
of the module, except if they are part of an input or output chain.

\Next
\Example
\begin{verbatim}

/* Copy the curve of a morpho line into another morpho line */
Morpho_line in,out=NULL;

out=mw_change_morpho_line(out);
if (!out) mwerror(FATAL,1,"Not enough memory !\n");
out->open = in->open;
if ( ((out->first_point = mw_new_point_curve()) == NULL) ||
     ((out->first_type = mw_new_point_type()) == NULL) )
   mwerror(FATAL, 1,"Not enough memory !\n");
mw_copy_point_curve(in->first_point,out->first_point);
mw_copy_point_type(in->first_type,out->first_type);

\end{verbatim}

\newpage %......................................


\Name{mw\_copy\_morpho\_line}{Copy a morpho line into another one}
\Summary{
Morpho\_line mw\_copy\_morpho\_line(in,out)

Morpho\_line in, out;
}
\Description
This function copies the \mline\ \verb+in+ into \verb+out+.
All fields are copied but the following : \verb+pdata+, \verb+morphosets+,
\verb+num+, \verb+previous+ and \verb+next+.
The result is put in \verb+out+, which may not be a predefined structure : in case 
of \verb+out=NULL+, the \verb+out+ structure is allocated.

The function \verb+mw_copy_morpho_line+ returns \Null\ if not enough memory is available to perform
the copy, or \verb+out+ elsewhere.
Your code should check this return value to send an error message in the \Null\ case, and do appropriate statement.

\Next
\Example
\begin{verbatim}

Morpho_line in; /* Predefined morpho_line */
Morpho_line out=NULL; 

out=mw_copy_morpho_line(in,out);
if (!out) mwerror(FATAL,1,"Not enough memory.\n");
\end{verbatim}

\newpage %......................................


\Name{mw\_delete\_morpho\_line}{Deallocate a morpho line}
\Summary{
void mw\_delete\_morpho\_line(ml)

Morpho\_line ml;
}
\Description
This function deallocates the \mline\ \verb+ml+ structure, including the
curve (\point) and the chain of types (\pt). Other pointers are
not deallicated.
You should line \verb+ml = NULL+ after this call since the address pointed
by \verb+ml+ is no longer valid.

\Next
\Example
\begin{verbatim}

Morpho_line ml; /* Internal use: no Input neither Output of module */
Point_curve pt;

/* Define a morpho line containing the point (0,0) only */

if (!(pt=mw_new_point_curve()) ||
    !(ml=mw_new_morpho_line())) mwerror(FATAL,1,"Not enough memory.\n");
pt->x=pt->y=0;
ml->first_point=pt;

/* .
   .
   (statement)
   .
   .
*/

/* Deallocate the morpho_line */
mw_delete_morpho_line(ml);

\end{verbatim}

\newpage %......................................


\Name{mw\_length\_morpho\_line}{Return the number of points a morpho line contains}
\Summary{
unsigned int mw\_length\_morpho\_line(ml)

Morpho\_line ml;
}
\Description
This function returns the number of points contained in the input
\verb+ml+.
It returns $0$ if the structure is empty or undefined.
If the field \verb+first_type+ is not \Null, the number of points
defined by this field must equal the number of points in the curve.

\Next
\Example
\begin{verbatim}

Morpho_line ml; /* Internal use: no Input neither Output of module */
Point_curve pt;

/* Define a morpho line containing the point (0,0) only */

if (!(pt=mw_new_point_curve()) ||
    !(ml=mw_new_morpho_line())) mwerror(FATAL,1,"Not enough memory.\n");
pt->x=pt->y=0;
ml->first_point=pt;

/* This will print 1 */
printf("%d",mw_length_morpho_line(ml));
\end{verbatim}

\newpage %......................................


\Name{mw\_new\_morpho\_line}{Create a new morpho line}
\Summary{
Morpho\_line mw\_new\_morpho\_line()

}
\Description
This function returns a new \mline\ structure, or \Null\ if not enough 
memory is available to allocate the structure. 
Your code should check this return value to send an error message in the 
\Null\ case, and do appropriate statement.

The new structure is created with fields set to $0$ or \Null.

\Next
\Example
\begin{verbatim}

/* Copy the curve of a morpho line into another morpho line */
Morpho_line in,out;

out=mw_new_morpho_line();
if (!out) mwerror(FATAL,1,"Not enough memory !\n");
out->open = in->open;
if ( ((out->first_point = mw_new_point_curve()) == NULL) ||
     ((out->first_type = mw_new_point_type()) == NULL) )
   mwerror(FATAL, 1,"Not enough memory !\n");
mw_copy_point_curve(in->first_point,out->first_point);
mw_copy_point_type(in->first_type,out->first_type);

\end{verbatim}

\newpage %......................................


