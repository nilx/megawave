\Name{mw\_change\_mimage}{Define a morphological image, if not already defined}
\Summary{
Mimage mw\_change\_mimage(mi)

Mimage mi;
}
\Description
This function returns a \mimage\ structure if the input \verb+mi = NULL+.
It is provided despite the \\
\verb+mw_new_mimage()+ function for
global coherence with other memory types.

The function \verb+mw_change_mimage+ returns \Null\ if not enough memory is available to allocate the structure. 
Your code should check this return value to send an error message in the \Null\ case, and do appropriate statement.

Since the MegaWave2 compiler allocates structures for input and output 
objects (see \volI), this function is normally used only for internal objects.
Do not forget to deallocate the internal structures before the end
of the module, except if they are part of an input or output chain.

\Next
\Example
\begin{verbatim}
/* Copy the morpho lines only of a morphological image into another morphological image */

Mimage in,out=NULL;

out=mw_change_mimage(out);
if (!out) mwerror(FATAL,1,"Not enough memory !\n");
out->nrow = in->nrow;
out->ncol = in->ncol;
out->minvalue=in->minvalue;
out->maxvalue=in->maxvalue;
if (in->firstml) 
  {
    out->firstml=mw_copy_morpho_line(in->firstml, out->firstml);
    if (!out->firstml) mwerror(FATAL, 1,"Not enough memory !\n");
  }
\end{verbatim}

\newpage %......................................


\Name{mw\_copy\_mimage}{Copy a morphological image into another one}
\Summary{
Mimage mw\_copy\_mimage(in,out)

Mimage in, out;
}
\Description
This function copies the \mimage\ \verb+in+ into \verb+out+.
All fields are copied, including the chains of \msets, \mline\ and \fmline.
The result is put in \verb+out+, which may not be a predefined structure : in case 
of \verb+out=NULL+, the \verb+out+ structure is allocated.

The function \verb+mw_copy_mimage+ returns \Null\ if not enough memory is available to perform
the copy, or \verb+out+ elsewhere.
Your code should check this return value to send an error message in the \Null\ case, and do appropriate statement.

\Next
\Example
\begin{verbatim}

mimage in; /* Predefined mimage */
mimage out=NULL; 

out=mw_copy_mimage(in,out);
if (!out) mwerror(FATAL,1,"Not enough memory.\n");
\end{verbatim}

\newpage %......................................


\Name{mw\_delete\_mimage}{Deallocate a morphological image}
\Summary{
void mw\_delete\_mimage(mi)

mimage mi;
}
\Description
This function deallocates the \mimage\ \verb+mi+ structure, including the
chains of \msets, \mline\ and \fmline.
You should line \verb+mi = NULL+ after this call since the address pointed
by \verb+mi+ is no longer valid.

\Next
\Example
\begin{verbatim}

Mimage mi; /* Internal use: no Input neither Output of module */

Morpho_line ml; 
Point_curve pt;
Fmorpho_line fml; 
Point_fcurve fpt;
Morpho_sets mss;
Hsegment seg;
Morpho_set ms;


/* Define a morpho line containing the point (0,0) only */

if (!(pt=mw_new_point_curve()) ||
    !(ml=mw_new_morpho_line())) mwerror(FATAL,1,"Not enough memory.\n");
pt->x=pt->y=0;
ml->first_point=pt;

/* Define a fmorpho line containing the point (0.5,0.5) only */

if (!(fpt=mw_new_point_fcurve()) ||
    !(fml=mw_new_fmorpho_line())) mwerror(FATAL,1,"Not enough memory.\n");
fpt->x=fpt->y=0.5;
fml->first_point=fpt;

/* Define a morpho sets containing one morpho set */

if (!(seg=mw_new_hsegment()) ||
    !(ms=mw_new_morpho_set()) ||
    !(mss=mw_new_morpho_sets())) mwerror(FATAL,1,"Not enough memory.\n");
seg->xstart=0; 
seg->xend=200;
seg->y=10;
ms->first_segment=seg; ms->minvalue=0.0; ms->maxvalue = 1.0; ms->area=201;
mss->morphoset=ms;

/* Define a morphological image made by one morpho line, one fmorpho line and
   one morpho sets.
*/

if (!(mi=mw_new_morpho_line())) mwerror(FATAL,1,"Not enough memory.\n");
mi->first_ml=ml;
mi->first_fml=fml;
mi->first_ms=ms;

/* .
   .
   (statement)
   .
   .
*/


/* Deallocate the mimage, including ml, fml and ms */
mw_delete_mimage(mi);

\end{verbatim}

\newpage %......................................


\Name{mw\_length\_fml\_mimage}{Return the number of morpho lines a morphological image contains}
\Summary{
unsigned int mw\_length\_fml\_mimage(mi)

Mimage mi;
}
\Description
This function returns the number of fmorpho lines contained in the input
\verb+mi+.
It returns $0$ if the structure is empty or undefined.

\Next
\Example
See example page~\pageref{length_ml-example}.

\newpage %......................................


\Name{mw\_length\_ml\_mimage}{Return the number of morpho lines a morphological image contains}
\Summary{
unsigned int mw\_length\_ml\_mimage(mi)

Mimage mi;
}
\Description
This function returns the number of morpho lines contained in the input
\verb+mi+.
It returns $0$ if the structure is empty or undefined.

\Next
\Example
\label{length_ml-example}
\begin{verbatim}


Mimage mi; /* Internal use: no Input neither Output of module */

Morpho_line ml; 
Point_curve pt;
Fmorpho_line fml; 
Point_fcurve fpt;
Morpho_sets mss;
Hsegment seg;
Morpho_set ms;


/* Define a morpho line containing the point (0,0) only */

if (!(pt=mw_new_point_curve()) ||
    !(ml=mw_new_morpho_line())) mwerror(FATAL,1,"Not enough memory.\n");
pt->x=pt->y=0;
ml->first_point=pt;

/* Define a fmorpho line containing the point (0.5,0.5) only */

if (!(fpt=mw_new_point_fcurve()) ||
    !(fml=mw_new_fmorpho_line())) mwerror(FATAL,1,"Not enough memory.\n");
fpt->x=fpt->y=0.5;
fml->first_point=fpt;

/* Define a morpho sets containing one morpho set */

if (!(seg=mw_new_hsegment()) ||
    !(ms=mw_new_morpho_set()) ||
    !(mss=mw_new_morpho_sets())) mwerror(FATAL,1,"Not enough memory.\n");
seg->xstart=0; 
seg->xend=200;
seg->y=10;
ms->first_segment=seg; ms->minvalue=0.0; ms->maxvalue = 1.0; ms->area=201;
mss->morphoset=ms;

/* Define a morphological image made by one morpho line, one fmorpho line and
   one morpho sets.
*/

if (!(mi=mw_new_morpho_line())) mwerror(FATAL,1,"Not enough memory.\n");
mi->first_ml=ml;
mi->first_fml=fml;
mi->first_ms=ms;


/* This will print 1 */
printf("%d",mw_length_ml_mimage(mi));
/* This will print 1 */
printf("%d",mw_length_fml_mimage(mi));
/* This will print 1 */
printf("%d",mw_length_ms_mimage(mi));

\end{verbatim}

\newpage %......................................


\Name{mw\_length\_ms\_mimage}{Return the number of morpho sets a morphological image contains}
\Summary{
unsigned int mw\_length\_ms\_mimage(mi)

Mimage mi;
}
\Description
This function returns the number of morpho sets contained in the input
\verb+mi+.
It returns $0$ if the structure is empty or undefined.

\Next
\Example
See example page~\pageref{length_ml-example}.

\newpage %......................................


\Name{mw\_new\_mimage}{Create a new morphological image}
\Summary{
Mimage mw\_new\_mimage()

}
\Description
This function returns a new \mimage\ structure, or \Null\ if not enough 
memory is available to allocate the structure. 
Your code should check this return value to send an error message in the 
\Null\ case, and do appropriate statement.

The new structure is created with fields set to $0$ or \Null.

\Next
\Example
\begin{verbatim}
/* Copy the morpho lines only of a morphological image into another morphological image */

Mimage in,out;

out=mw_new_mimage();
if (!out) mwerror(FATAL,1,"Not enough memory !\n");
out->nrow = in->nrow;
out->ncol = in->ncol;
out->minvalue=in->minvalue;
out->maxvalue=in->maxvalue;
if (in->firstml) 
  {
    out->firstml=mw_copy_morpho_line(in->firstml, out->firstml);
    if (!out->firstml) mwerror(FATAL, 1,"Not enough memory !\n");
  }
\end{verbatim}

\newpage %......................................


