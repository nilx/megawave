\Name{mw\_change\_polygons}{Define the polygons structure, if not defined}
\Summary{
Polygons mw\_change\_polygons(polygons)

Polygons polygons;
}
\Description
This function returns a Polygons structure if the input \verb+polygons = NULL+.
It is provided despite the \verb+mw_new_polygons+ function for
global coherence with other memory types.

The function \verb+mw_change_polygons+ returns \Null\ if not enough memory is available to allocate the structure. 
Your code should check this return value to send an error message in the \Null\ case, and do appropriate statement.

Since the MegaWave2 compiler allocates structures for input and output 
objects (see \volI), this function is normally used only for internal objects.
Do not forget to deallocate the internal polygons structures before the end
of the module.

\Next
\Example
\begin{verbatim}
/* Define a polygons set to be two pre-defined polygons  */

Polygons polygons=NULL;   /* Internal use: no Input neither Output of module */
Polygon polygon1,polygon2;  /* Pre-defined polygons (e.g. inputs of module) */

polygons = mw_change_polygons(polygons);
if (polygons == NULL) mwerror(FATAL,1,"Not enough memory.\n");
...
\end{verbatim}
(End of this example as for the \verb+mw_new_polygons+ function).


\newpage %......................................


\Name{mw\_delete\_polygons}{Deallocate a polygons set}
\Summary{
void mw\_delete\_polygons(polygons)

Polygons polygons;
}
\Description
This function deallocates all the memory allocated by the polygons variable
that is, all the points belonging to all polygons into this set, all channels arrays (if any), all \polygon structures and the \polygons structure itself.
You should set \verb+polygons = NULL+ after this call since the address pointed
by \verb+polygons+ is no longer valid.

\Next
\Example
\begin{verbatim}
Polygons polygons=NULL;   /* Internal use: no Input neither Output of module */

polygons = mw_new_polygons();
if (polygons == NULL) mwerror(FATAL,1,"Not enough memory.\n");
.
.
.
mw_delete_polygons(polygons);
\end{verbatim}

\newpage %......................................


\Name{mw\_length\_polygons}{Return the number of polygons into a polygons structure}
\Summary{
unsigned int mw\_length\_polygons(polys);

Polygons polys;
}
\Description
This function returns the number of polygons contained in the given
\verb+polys+. It returns $0$ if the structure is empty.

\Next
\Example
\begin{verbatim}
/* Define a polygons set to be two pre-defined polygons  */

Polygons polygons=NULL;   /* Internal use: no Input neither Output of module */
Polygon polygon1,polygon2;  /* Pre-defined polygons (e.g. inputs of module) */

polygons = mw_new_polygons();
if (polygons == NULL) mwerror(FATAL,1,"Not enough memory.\n");

polygons->first=polygon1;
polygon1->previous = polygon2->next = NULL;
polygon1->next = polygon2;
polygon2->previous = polygon1;

/* The length would be 2 */
printf("Length=%d\n",mw_length_polygons(polygons));

\end{verbatim}
\newpage %......................................

\Name{mw\_new\_polygons}{Create a new polygons}
\Summary{
Polygons mw\_new\_polygons();
}
\Description
This function creates a new \polygons structure.
It returns \Null\ if not enough memory is available to create the structure.
Your code should check this value to send an
error message in the \Null\ case, and do appropriate statement.

Since the MegaWave2 compiler allocates structures for input and output 
objects (see \volI), this function is normally used only for internal objects.
Do not forget to deallocate the internal structures before the end
of the module.

\Next
\Example
\begin{verbatim}
/* Define a polygons set to be two pre-defined polygons  */

Polygons polygons=NULL;   /* Internal use: no Input neither Output of module */
Polygon polygon1,polygon2;  /* Pre-defined polygons (e.g. inputs of module) */

polygons = mw_new_polygons();
if (polygons == NULL) mwerror(FATAL,1,"Not enough memory.\n");

polygons->first=polygon1;
polygon1->previous = polygon2->next = NULL;
polygon1->next = polygon2;
polygon2->previous = polygon1;
\end{verbatim}
\newpage %......................................

