\Name{mw\_change\_point\_curve}{Define the point\_curve structure, if not defined}
\Summary{
Point\_curve mw\_change\_point\_curve(point)

Point\_curve point;
}
\Description
This function returns a Point\_curve structure if the input \verb+point = NULL+.
It is provided despite the \verb+mw_new_point_curve+ function for
global coherence with other memory types.

The function \verb+mw_change_point_curve+ returns \Null\ if not enough memory is available to allocate the structure. 
Your code should check this return value to send an error message in the \Null\ case, and do appropriate statement.

Since the MegaWave2 compiler allocates structures for input and output 
objects (see \volI), this function is normally used only for internal objects.
Do not forget to deallocate the internal structures before the end
of the module, except if they are part of an input or output chain.

\Next
\Example
\begin{verbatim}

Point_curve point=NULL; /* Internal use: no Input neither Output of module */

/* Define the point (5,1) of the plane */

point = mw_change_point_curve(point);
if (point == NULL) mwerror(FATAL,1,"Not enough memory.\n");
point->x = 5;
point->y = 1;
\end{verbatim}

\newpage %......................................


\Name{mw\_copy\_point\_curve}{Copy all points starting from the given one}
\Summary{
Point\_curve mw\_copy\_point\_curve(in,out)

Point\_curve in, out;
}
\Description
This function copies the current point and the next points contained in the chain 
defined at the starting point \verb+in+. The result is put in \verb+out+, which
may not be a predefined structure : in case of \verb+out=NULL+, the \verb+out+
structure is allocated.

The function \verb+mw_copy_point_curve+ returns \Null\ if not enough memory is available to perform
the copy, or \verb+out+ elsewhere.
Your code should check this return value to send an error message in the \Null\ case, and do appropriate statement.

\Next
\Example
\begin{verbatim}

Point_curve in; /* Predefined point */
Point_curve out=NULL; 

out=mw_copy_point_curve(in,out);
if (!out) mwerror(FATAL,1,"Not enough memory.\n");
\end{verbatim}

\newpage %......................................


\Name{mw\_delete\_point\_curve}{Deallocate the point\_curve structure}
\Summary{
void mw\_delete\_point\_curve(point)

Point\_curve point;
}
\Description
This function deallocates the \point\ structures starting from
the given \verb+point+, including this point itself.
You should set \verb+point = NULL+ after this call since the address pointed
by \verb+point+ is no longer valid.
Warning : to deallocate only a point and not all the next points of a 
chain, just use \verb+free(point)+.

\Next
\Example
\begin{verbatim}
/* Remove the first point of an existing curve */

Curve curve;  /* Existing curve (e.g. Input of module) */
Point_curve point;  /* Internal use */

point = curve->first;
curve->first=point->next;
point->next->previous = NULL;
free(point);
point = NULL;

/* Remove all points of an existing curve */

mw_delete_point_curve(curve->first);
\end{verbatim}

\newpage %......................................


\Name{mw\_new\_point\_curve}{Create a new point\_curve structure}
\Summary{
Point\_curve mw\_new\_point\_curve();
}
\Description
This function creates a new \point\ structure.
It returns \Null\ if not enough memory is available to create the structure.
Your code should check this value to send an
error message in the \Null\ case, and do appropriate statement.

Since the MegaWave2 compiler allocates structures for input and output 
objects (see \volI), this function is normally used only for internal objects.
Do not forget to deallocate the internal point structures before the end
of the module, except if they are part of an input or output curve.

\Next
\Example
\begin{verbatim}
/* Insert the point (0,0) at the end of an existing curve */

Curve curve;  /* Existing curve (e.g. Input of module) */
Point_curve point,p;  /* Internal use: no Input neither Output of module */

/* Define the point (0,0) */
point = mw_new_point_curve();
if (point == NULL) mwerror(FATAL,1,"Not enough memory.\n");
point->x = point->y = 0;
point->next = NULL;

/* Find the last point of the curve */
p = curve->first; while (p->next) p=p->next;

/* Insert the point */
p->next = point;
point->previous = p;

/* Do not deallocate point or curve will become inconsistent */ 
\end{verbatim}
\newpage %......................................

