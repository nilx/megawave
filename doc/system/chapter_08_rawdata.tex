\Name{mw\_alloc\_rawdata}{Allocate the data array of a \rawdata\ structure}
\Summary{
Rawdata mw\_alloc\_rawdata(rd,size)

Rawdata rd;

int size;
}
\Description
This function allocates the \verb+data+ array of a \rawdata structure previously created using \verb+mw_new_rawdata+. The size of the data is given by 
\verb+size+, it corresponds to the number of bytes.

Values can be addressed after this call, if the allocation successed. There is
no default values.

Do not use this function if \verb+rd+ has already an allocated array: use
the function \verb+mw_change_rawdata+ instead.

The function \verb+mw_alloc_rawdata+ returns \Null\ if not enough memory is available to allocate the array. Your code should check this return value to 
send an error message in the \Null\ case, and do appropriate statement.

\Next
\Example
\begin{verbatim}
Rawdata rd=NULL; /* Internal use: no Input neither Output of module */
int i;

/* Create a rawdata of 1000 bytes */
if  ( ((rd = mw_new_rawdata()) == NULL) ||
      (mw_alloc_rawdata(rd,1000) == NULL) )
    mwerror(FATAL,1,"Not enough memory.\n");

/* Set the byte #i to the value i mod 256 */
for (i=0;i<rd->size;i++) rd->data[i] = i % 256;
\end{verbatim}

\newpage %......................................

\Name{mw\_change\_rawdata}{Change the size of the data  array of a \rawdata\ structure}
\Summary{
Rawdata mw\_change\_rawdata(rd, newsize)

Rawdata rd;

int newsize;
}
\Description
This function changes the memory allocation of the \verb+data+ array of a \rawdata\ structure, even if no previously memory allocation was done.
The new size of the array is given by \verb+newsize+, it corresponds to the number 
of allocated bytes.

The function \verb+mw_change_rawdata+ can also create the structure if the input \verb+rd = NULL+.
Therefore, this function can replace both \verb+mw_new_rawdata+ and
\verb+mw_alloc_rawdata+. It is the recommended function to allocate \rawdata\ variables
used as input/output of modules. Since the function can set the 
address of \verb+rd+, the variable must be set to the return value of 
the function (See example below).

The function \verb+mw_change_rawdata+ returns \Null\ if not enough memory is available to allocate the array. 
Your code should check this return value to 
send an error message in the \Null\ case, and do appropriate statement.

\Next
\Example
\begin{verbatim}

Rawdata Output; /* Output of module */

/* Set the size of the array to be 1000 bytes */
Output = mw_change_rawdata(Output, 1000);
if (Output == NULL) mwerror(FATAL,1,"Not enough memory.\n");
\end{verbatim}

\newpage %......................................

\Name{mw\_copy\_rawdata}{Copy the data of a \rawdata+ structure into another one}
\Summary{
void mw\_copy\_rawdata(in, out)

Rawdata in,out;
}
\Description
This function copies the content of the array \verb+data+ of the \rawdata+ structure \verb+in+ into the array \verb+data+ of \verb+out+.
The variable \verb+out+ must be an allocated \rawdata\ structure of same
size than \verb+in+.

The speed of this function depends to the C library implementation, but it is
usually very fast (trying to do faster is a waste of time).

\Next
\Example
\begin{verbatim}

Rawdata G; /* Needed Input */
Rawdata F; /* Optional Output */

  if (F) {
    printf("F option is active: copy G in F\n");
    if ((F = mw_change_rawdata(F, G->size)) == NULL)
        mwerror(FATAL,1,"Not enough memory.\n");
    else mw_copy_rawdata(G, F);
   }
  else  printf("F option is not active\n");
\end{verbatim}

\newpage %......................................

\Name{mw\_delete\_rawdata}{Deallocate the data array of a \rawdata\ structure}
\Summary{
void mw\_delete\_rawdata(rd)

Rawdata rd;
}
\Description
This function deallocates the array \verb+values+ of a \rawdata\ structure previously allocated using\\
 \verb+mw_alloc_rawdata+ or \verb+mw_change_rawdata+, and the structure itself. 

You should set \verb+rd = NULL+ after this call since the address pointed
by \verb+rd+ is no longer valid.

\Next
\Example
\begin{verbatim}
Rawdata rd=NULL; /* Internal use: no Input neither Output of module */

if  ( ((rd = mw_new_rawdata()) == NULL) ||
      (mw_alloc_rawdata(rd,1000) == NULL) )
    mwerror(FATAL,1,"Not enough memory.\n");
/* .
   .
   (statement)
   .
   .
*/
mw_delete_rawdata(rd);
rd = NULL;

\end{verbatim}

\newpage %......................................


\Name{mw\_new\_rawdata}{Create a new \rawdata\ structure}
\Summary{
Rawdata mw\_new\_rawdata();
}
\Description
This function creates a new \rawdata\ structure with an empty \verb+data+ array and
\verb+size+ field set to $0$.
No data can be addressed at this time.
The \verb+data+ should  be allocated using the function \verb+mw_alloc_rawdata+ or \verb+mw_change_rawdata+.

Do not use this function for input/output of modules, since the MegaWave2
Compiler already created the structure for you if you need it (see \volI). Use instead the function \verb+mw_change_rawdata+.
Do not forget to deallocate the internal structures before the end
of the module.

The function \verb+mw_new_rawdata+ returns \Null\ if not enough memory is available to create the structure. Your code should check this value to send an
error message in the \Null\ case, and do appropriate statement.

\Next
\Example
\begin{verbatim}
Rawdata rd=NULL; /* Internal use: no Input neither Output of module */

if  ( ((rd = mw_new_rawdata()) == NULL) ||
      (mw_alloc_rawdata(rd,1000) == NULL) )
    mwerror(FATAL,1,"Not enough memory.\n");
\end{verbatim}

\newpage %......................................

