\Name{mw\_change\_curve}{Define the curve structure, if not defined}
\Summary{
Curve mw\_change\_curve(curve)

Curve curve;
}
\Description
This function returns a Curve structure if the input \verb+curve = NULL+.
It is provided despite the \verb+mw_new_curve+ function for
global coherence with other memory types.

The function \verb+mw_change_curve+ returns \Null\ if not enough memory is available to allocate the structure. 
Your code should check this return value to send an error message in the \Null\ case, and do appropriate statement.

Since the MegaWave2 compiler allocates structures for input and output 
objects (see \volI), this function is normally used only for internal objects.
Do not forget to deallocate the internal curve structures before the end
of the module, except if they are part of an input or output chain.

\Next
\Example
\begin{verbatim}
/* Define a curve with 10 points which is the straight line (0,0)-(9,9) */

Curve curve=NULL; /* Internal use: no Input neither Output of module */
Point_curve newp,oldp=NULL;
int i;

curve = mw_change_curve(curve);
if (curve == NULL) mwerror(FATAL,1,"Not enough memory.\n");
...
\end{verbatim}
(End of this example as for the \verb+mw_new_curve+ function).


\newpage %......................................


\Name{mw\_copy\_curve}{Copy a curve into another one}
\Summary{
Curve mw\_copy\_curve(in,out)

Curve in, out;
}
\Description
This function duplicates the points contained in \verb+in+.
The result is put in \verb+out+, which may not be a predefined structure : in case 
of \verb+out=NULL+, the \verb+out+ structure is allocated.

The function \verb+mw_copy_curve+ returns \Null\ if not enough memory is available to perform
the copy, or \verb+out+ elsewhere.
Your code should check this return value to send an error message in the \Null\ case, and do appropriate statement.

\Next
\Example
\begin{verbatim}

Curve in; /* Predefined curve */
Curve out=NULL; 

out=mw_copy_curve(in,out);
if (!out) mwerror(FATAL,1,"Not enough memory.\n");
\end{verbatim}

\newpage %......................................


\Name{mw\_delete\_curve}{Deallocate a curve}
\Summary{
void mw\_delete\_curve(curve)

Curve curve;
}
\Description
This function deallocates all the memory allocated by the curve variable that is, all the points belonging to this chain and the \curve structure itself.
You should set \verb+curve = NULL+ after this call since the address pointed
by \verb+curve+ is no longer valid.

\Next
\Example
\begin{verbatim}
/* Remove the first curve of an existing curve set (curves) */

Curves curves;/* Existing curve set (e.g. Input of module) */
Curve curve;  /* Internal use */

curve = curves->first;
curves->first=curves->next;
curves->next->previous = NULL;
mw_delete_curve(curve);
curve = NULL;
\end{verbatim}

\newpage %......................................


\Name{mw\_length\_curve}{Return the number of points of a curve}
\Summary{
unsigned int mw\_length\_curve(cv);

Curve cv;
}
\Description
This function return the number of points contained in the given
curve \verb+cv+. It returns $0$ if the structure is empty.

\Next
\Example
\begin{verbatim}

Curve curve; /* Internal use: no Input neither Output of module */
Point_curve newp,oldp=NULL;
int i;

curve = mw_new_curve();
if (curve == NULL) mwerror(FATAL,1,"Not enough memory.\n");

/* Define a curve with 5 points */
for (i=1;i<=5;i++)
{
 newp = mw_new_point_curve();
 if (newp == NULL) mwerror(FATAL,1,"Not enough memory.\n");
 if (i=0) curve->first = newp;
 newp->x = newp->y = i;
 newp->previous = oldp;
 if (oldp) oldp->next = newp;
 oldp=newp;
} 

/* The length is 5 */
printf("Length=%d\n",mw_length_curve(curve));

\end{verbatim}
\newpage %......................................

\Name{mw\_new\_curve}{Create a new curve}
\Summary{
Curve mw\_new\_curve();
}
\Description
This function creates a new \curve structure.
It returns \Null\ if not enough memory is available to create the structure.
Your code should check this value to send an
error message in the \Null\ case, and do appropriate statement.

Since the MegaWave2 compiler allocates structures for input and output 
objects (see \volI), this function is normally used only for internal objects.
Do not forget to deallocate the internal structures before the end
of the module, except if they are part of an input or output chain.

\Next
\Example
\begin{verbatim}
/* Define a curve with 10 points which is the straight line (0,0)-(9,9) */

Curve curve; /* Internal use: no Input neither Output of module */
Point_curve newp,oldp=NULL;
int i;

curve = mw_new_curve();
if (curve == NULL) mwerror(FATAL,1,"Not enough memory.\n");

for (i=0;i<10;i++)
{
 newp = mw_new_point_curve();
 if (newp == NULL) mwerror(FATAL,1,"Not enough memory.\n");
 if (i=0) curve->first = newp;
 newp->x = newp->y = i;
 newp->previous = oldp;
 if (oldp) oldp->next = newp;
 oldp=newp;
} 
\end{verbatim}
\newpage %......................................

